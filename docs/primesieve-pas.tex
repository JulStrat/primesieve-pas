% Generated by https://github.com/pasdoc/pasdoc/wikiPasDoc 0.15.0
\documentclass{report}
\usepackage{hyperref}
% WARNING: THIS SHOULD BE MODIFIED DEPENDING ON THE LETTER/A4 SIZE
\oddsidemargin 0cm
\evensidemargin 0cm
\marginparsep 0cm
\marginparwidth 0cm
\parindent 0cm
\setlength{\textwidth}{\paperwidth}
\addtolength{\textwidth}{-2in}


% Conditional define to determine if pdf output is used
\newif\ifpdf
\ifx\pdfoutput\undefined
\pdffalse
\else
\pdfoutput=1
\pdftrue
\fi

\ifpdf
  \usepackage[pdftex]{graphicx}
\else
  \usepackage[dvips]{graphicx}
\fi

% Write Document information for pdflatex/pdftex
\ifpdf
\pdfinfo{
 /Author     (I. Kakoulidis)
 /Title      (primesieve-pas)
}
\fi


\begin{document}
\title{primesieve-pas}
\author{I. Kakoulidis}
\maketitle
\newpage
\label{toc}\tableofcontents
\newpage
% special variable used for calculating some widths.
\newlength{\tmplength}
\chapter{Unit primesieve}
\label{primesieve}
\index{primesieve}
\section{Description}
Pascal bindings for primesieve library.

primesieve {-} library for fast prime number generation.\\{} Copyright (C) 2019 Kim Walisch, {$<$}kim.walisch@gmail.com{$>$}\\{} \href{https://github.com/kimwalisch/primesieve}{https://github.com/kimwalisch/primesieve}

primesieve{-}pas {-} FPC/Delphi API for primesieve library.\\{} Copyright (C) 2020 I. Kakoulidis, {$<$}ioulianos.kakoulidis@hotmail.com{$>$}\\{} \href{https://github.com/JulStrat/primesieve-pas}{https://github.com/JulStrat/primesieve-pas}

This file is distributed under the BSD 2{-}Clause License.
\section{Overview}
\begin{description}
\item[\texttt{\begin{ttfamily}primesieve{\_}iterator\end{ttfamily} Record}]
\end{description}
\begin{description}
\item[\texttt{primesieve{\_}generate{\_}primes}]
\item[\texttt{primesieve{\_}generate{\_}n{\_}primes}]
\item[\texttt{primesieve{\_}nth{\_}prime}]
\item[\texttt{primesieve{\_}count{\_}primes}]
\item[\texttt{primesieve{\_}count{\_}twins}]
\item[\texttt{primesieve{\_}count{\_}triplets}]
\item[\texttt{primesieve{\_}count{\_}quadruplets}]
\item[\texttt{primesieve{\_}count{\_}quintuplets}]
\item[\texttt{primesieve{\_}count{\_}sextuplets}]
\item[\texttt{primesieve{\_}print{\_}primes}]
\item[\texttt{primesieve{\_}print{\_}twins}]
\item[\texttt{primesieve{\_}print{\_}triplets}]
\item[\texttt{primesieve{\_}print{\_}quadruplets}]
\item[\texttt{primesieve{\_}print{\_}quintuplets}]
\item[\texttt{primesieve{\_}print{\_}sextuplets}]
\item[\texttt{primesieve{\_}get{\_}max{\_}stop}]
\item[\texttt{primesieve{\_}get{\_}sieve{\_}size}]
\item[\texttt{primesieve{\_}get{\_}num{\_}threads}]
\item[\texttt{primesieve{\_}set{\_}sieve{\_}size}]
\item[\texttt{primesieve{\_}set{\_}num{\_}threads}]
\item[\texttt{primesieve{\_}free}]
\item[\texttt{primesieve{\_}version}]
\item[\texttt{primesieve{\_}init}]
\item[\texttt{primesieve{\_}free{\_}iterator}]
\item[\texttt{primesieve{\_}skipto}]
\item[\texttt{primesieve{\_}next{\_}prime}]
\item[\texttt{primesieve{\_}prev{\_}prime}]
\end{description}
\section{Classes, Interfaces, Objects and Records}
\ifpdf
\subsection*{\large{\textbf{primesieve{\_}iterator Record}}\normalsize\hspace{1ex}\hrulefill}
\else
\subsection*{primesieve{\_}iterator Record}
\fi
\label{primesieve.primesieve_iterator}
\index{primesieve{\_}iterator}
\subsubsection*{\large{\textbf{Description}}\normalsize\hspace{1ex}\hfill}
\begin{ttfamily}primesieve{\_}iterator\end{ttfamily}(\ref{primesieve.primesieve_iterator}) allows to easily iterate over primes both forwards and backwards. Generating the first prime has a complexity of \textit{O(r log log r)} operations with \textit{r = n{\^{}}0.5}, after that any additional prime is generated in amortized \textit{O(log n log log n)} operations. The memory usage is about \textit{PrimePi(n{\^{}}0.5) * 8} bytes.

The \textit{primesieve{\_}iterator.pas} example shows how to use \begin{ttfamily}primesieve{\_}iterator\end{ttfamily}(\ref{primesieve.primesieve_iterator}). If any error occurs \begin{ttfamily}primesieve{\_}next{\_}prime\end{ttfamily}(\ref{primesieve-primesieve_next_prime}) and \begin{ttfamily}primesieve{\_}prev{\_}prime\end{ttfamily}(\ref{primesieve-primesieve_prev_prime}) return \begin{ttfamily}{\_}PRIMESIEVE{\_}ERROR\end{ttfamily}(\ref{primesieve-_PRIMESIEVE_ERROR}). Furthermore \textit{primesieve{\_}iterator.is{\_}error} is initialized to \textit{0} and set to \textit{1} if any error occurs.\section{Functions and Procedures}
\ifpdf
\subsection*{\large{\textbf{primesieve{\_}generate{\_}primes}}\normalsize\hspace{1ex}\hrulefill}
\else
\subsection*{primesieve{\_}generate{\_}primes}
\fi
\label{primesieve-primesieve_generate_primes}
\index{primesieve{\_}generate{\_}primes}
\begin{list}{}{
\settowidth{\tmplength}{\textbf{Description}}
\setlength{\itemindent}{0cm}
\setlength{\listparindent}{0cm}
\setlength{\leftmargin}{\evensidemargin}
\addtolength{\leftmargin}{\tmplength}
\settowidth{\labelsep}{X}
\addtolength{\leftmargin}{\labelsep}
\setlength{\labelwidth}{\tmplength}
}
\item[\textbf{Declaration}\hfill]
\ifpdf
\begin{flushleft}
\fi
\begin{ttfamily}
function primesieve{\_}generate{\_}primes(start: UInt64; stop: UInt64; var size: NativeUInt; ptype: Integer): Pointer; cdecl; external LIB{\_}PRIMESIEVE name LIB{\_}FNPFX + 'primesieve{\_}generate{\_}primes';\end{ttfamily}

\ifpdf
\end{flushleft}
\fi

\par
\item[\textbf{Description}]
Get an array with the primes inside the interval \textit{[start, stop]}.

 \par
\item[\textbf{Parameters}]
\begin{description}
\item[size] The size of the returned primes array
\item[ptype] The type of the primes to generate, e.g. \begin{ttfamily}INT{\_}PRIMES32\end{ttfamily}
\end{description}


\end{list}
\ifpdf
\subsection*{\large{\textbf{primesieve{\_}generate{\_}n{\_}primes}}\normalsize\hspace{1ex}\hrulefill}
\else
\subsection*{primesieve{\_}generate{\_}n{\_}primes}
\fi
\label{primesieve-primesieve_generate_n_primes}
\index{primesieve{\_}generate{\_}n{\_}primes}
\begin{list}{}{
\settowidth{\tmplength}{\textbf{Description}}
\setlength{\itemindent}{0cm}
\setlength{\listparindent}{0cm}
\setlength{\leftmargin}{\evensidemargin}
\addtolength{\leftmargin}{\tmplength}
\settowidth{\labelsep}{X}
\addtolength{\leftmargin}{\labelsep}
\setlength{\labelwidth}{\tmplength}
}
\item[\textbf{Declaration}\hfill]
\ifpdf
\begin{flushleft}
\fi
\begin{ttfamily}
function primesieve{\_}generate{\_}n{\_}primes(n: UInt64; start: UInt64; ptype: Integer): Pointer; cdecl; external LIB{\_}PRIMESIEVE name LIB{\_}FNPFX + 'primesieve{\_}generate{\_}n{\_}primes';\end{ttfamily}

\ifpdf
\end{flushleft}
\fi

\par
\item[\textbf{Description}]
Get an array with the first \textit{n primes {$>$}= start}.

\par
\item[\textbf{Parameters}]
\begin{description}
\item[ptype] The type of the primes to generate, e.g. \begin{ttfamily}INT{\_}PRIMES32\end{ttfamily}
\end{description}


\end{list}
\ifpdf
\subsection*{\large{\textbf{primesieve{\_}nth{\_}prime}}\normalsize\hspace{1ex}\hrulefill}
\else
\subsection*{primesieve{\_}nth{\_}prime}
\fi
\label{primesieve-primesieve_nth_prime}
\index{primesieve{\_}nth{\_}prime}
\begin{list}{}{
\settowidth{\tmplength}{\textbf{Description}}
\setlength{\itemindent}{0cm}
\setlength{\listparindent}{0cm}
\setlength{\leftmargin}{\evensidemargin}
\addtolength{\leftmargin}{\tmplength}
\settowidth{\labelsep}{X}
\addtolength{\leftmargin}{\labelsep}
\setlength{\labelwidth}{\tmplength}
}
\item[\textbf{Declaration}\hfill]
\ifpdf
\begin{flushleft}
\fi
\begin{ttfamily}
function primesieve{\_}nth{\_}prime(n: Int64; start: UInt64): UInt64; cdecl; external LIB{\_}PRIMESIEVE name LIB{\_}FNPFX + 'primesieve{\_}nth{\_}prime';\end{ttfamily}

\ifpdf
\end{flushleft}
\fi

\par
\item[\textbf{Description}]
Find the nth prime. By default all CPU cores are used, use \begin{ttfamily}primesieve{\_}set{\_}num{\_}threads\end{ttfamily}(\ref{primesieve-primesieve_set_num_threads}) to change the number of threads.

Note that each call to \begin{ttfamily}primesieve{\_}nth{\_}prime\end{ttfamily}(\ref{primesieve-primesieve_nth_prime}) incurs an initialization overhead of \textit{O(sqrt(start))} even if \textit{n} is tiny. Hence it is not a good idea to use \begin{ttfamily}primesieve{\_}nth{\_}prime\end{ttfamily}(\ref{primesieve-primesieve_nth_prime}) repeatedly in a loop to get the next (or previous) prime. For this use case it is better to use a \begin{ttfamily}primesieve{\_}iterator\end{ttfamily}(\ref{primesieve.primesieve_iterator}) which needs to be initialized only once.

\par
\item[\textbf{Parameters}]
\begin{description}
\item[n] if \textit{n = 0} finds the \textit{1st prime {$>$}= start},\\{} if \textit{n {$>$} 0} finds the \textit{nth prime {$>$} start},\\{} if \textit{n {$<$} 0} finds the \textit{nth prime {$<$} start} (backwards).
\end{description}


\end{list}
\ifpdf
\subsection*{\large{\textbf{primesieve{\_}count{\_}primes}}\normalsize\hspace{1ex}\hrulefill}
\else
\subsection*{primesieve{\_}count{\_}primes}
\fi
\label{primesieve-primesieve_count_primes}
\index{primesieve{\_}count{\_}primes}
\begin{list}{}{
\settowidth{\tmplength}{\textbf{Description}}
\setlength{\itemindent}{0cm}
\setlength{\listparindent}{0cm}
\setlength{\leftmargin}{\evensidemargin}
\addtolength{\leftmargin}{\tmplength}
\settowidth{\labelsep}{X}
\addtolength{\leftmargin}{\labelsep}
\setlength{\labelwidth}{\tmplength}
}
\item[\textbf{Declaration}\hfill]
\ifpdf
\begin{flushleft}
\fi
\begin{ttfamily}
function primesieve{\_}count{\_}primes(start: UInt64; stop: UInt64): UInt64; cdecl; external LIB{\_}PRIMESIEVE name LIB{\_}FNPFX + 'primesieve{\_}count{\_}primes';\end{ttfamily}

\ifpdf
\end{flushleft}
\fi

\par
\item[\textbf{Description}]
Count the primes within the interval \textit{[start, stop]}. By default all CPU cores are used, use \begin{ttfamily}primesieve{\_}set{\_}num{\_}threads\end{ttfamily}(\ref{primesieve-primesieve_set_num_threads}) to change the number of threads.

Note that each call to \begin{ttfamily}primesieve{\_}count{\_}primes\end{ttfamily}(\ref{primesieve-primesieve_count_primes}) incurs an initialization overhead of \textit{O(sqrt(stop))} even if the interval \textit{[start, stop]} is tiny. Hence if you have written an algorithm that makes many calls to \begin{ttfamily}primesieve{\_}count{\_}primes\end{ttfamily}(\ref{primesieve-primesieve_count_primes}) it may be preferable to use a \begin{ttfamily}primesieve{\_}iterator\end{ttfamily}(\ref{primesieve.primesieve_iterator}) which needs to be initialized only once.

\end{list}
\ifpdf
\subsection*{\large{\textbf{primesieve{\_}count{\_}twins}}\normalsize\hspace{1ex}\hrulefill}
\else
\subsection*{primesieve{\_}count{\_}twins}
\fi
\label{primesieve-primesieve_count_twins}
\index{primesieve{\_}count{\_}twins}
\begin{list}{}{
\settowidth{\tmplength}{\textbf{Description}}
\setlength{\itemindent}{0cm}
\setlength{\listparindent}{0cm}
\setlength{\leftmargin}{\evensidemargin}
\addtolength{\leftmargin}{\tmplength}
\settowidth{\labelsep}{X}
\addtolength{\leftmargin}{\labelsep}
\setlength{\labelwidth}{\tmplength}
}
\item[\textbf{Declaration}\hfill]
\ifpdf
\begin{flushleft}
\fi
\begin{ttfamily}
function primesieve{\_}count{\_}twins(start: UInt64; stop: UInt64): UInt64; cdecl; external LIB{\_}PRIMESIEVE name LIB{\_}FNPFX + 'primesieve{\_}count{\_}twins';\end{ttfamily}

\ifpdf
\end{flushleft}
\fi

\par
\item[\textbf{Description}]
Count the twin primes within the interval \textit{[start, stop]}.

By default all CPU cores are used, use \begin{ttfamily}primesieve{\_}set{\_}num{\_}threads\end{ttfamily}(\ref{primesieve-primesieve_set_num_threads}) to change the number of threads.

\end{list}
\ifpdf
\subsection*{\large{\textbf{primesieve{\_}count{\_}triplets}}\normalsize\hspace{1ex}\hrulefill}
\else
\subsection*{primesieve{\_}count{\_}triplets}
\fi
\label{primesieve-primesieve_count_triplets}
\index{primesieve{\_}count{\_}triplets}
\begin{list}{}{
\settowidth{\tmplength}{\textbf{Description}}
\setlength{\itemindent}{0cm}
\setlength{\listparindent}{0cm}
\setlength{\leftmargin}{\evensidemargin}
\addtolength{\leftmargin}{\tmplength}
\settowidth{\labelsep}{X}
\addtolength{\leftmargin}{\labelsep}
\setlength{\labelwidth}{\tmplength}
}
\item[\textbf{Declaration}\hfill]
\ifpdf
\begin{flushleft}
\fi
\begin{ttfamily}
function primesieve{\_}count{\_}triplets(start: UInt64; stop: UInt64): UInt64; cdecl; external LIB{\_}PRIMESIEVE name LIB{\_}FNPFX + 'primesieve{\_}count{\_}triplets';\end{ttfamily}

\ifpdf
\end{flushleft}
\fi

\par
\item[\textbf{Description}]
Count the prime triplets within the interval \textit{[start, stop]}.

By default all CPU cores are used, use \begin{ttfamily}primesieve{\_}set{\_}num{\_}threads\end{ttfamily}(\ref{primesieve-primesieve_set_num_threads}) to change the number of threads.

\end{list}
\ifpdf
\subsection*{\large{\textbf{primesieve{\_}count{\_}quadruplets}}\normalsize\hspace{1ex}\hrulefill}
\else
\subsection*{primesieve{\_}count{\_}quadruplets}
\fi
\label{primesieve-primesieve_count_quadruplets}
\index{primesieve{\_}count{\_}quadruplets}
\begin{list}{}{
\settowidth{\tmplength}{\textbf{Description}}
\setlength{\itemindent}{0cm}
\setlength{\listparindent}{0cm}
\setlength{\leftmargin}{\evensidemargin}
\addtolength{\leftmargin}{\tmplength}
\settowidth{\labelsep}{X}
\addtolength{\leftmargin}{\labelsep}
\setlength{\labelwidth}{\tmplength}
}
\item[\textbf{Declaration}\hfill]
\ifpdf
\begin{flushleft}
\fi
\begin{ttfamily}
function primesieve{\_}count{\_}quadruplets(start: UInt64; stop: UInt64): UInt64; cdecl; external LIB{\_}PRIMESIEVE name LIB{\_}FNPFX + 'primesieve{\_}count{\_}quadruplets';\end{ttfamily}

\ifpdf
\end{flushleft}
\fi

\par
\item[\textbf{Description}]
Count the prime quadruplets within the interval \textit{[start, stop]}.

By default all CPU cores are used, use \begin{ttfamily}primesieve{\_}set{\_}num{\_}threads\end{ttfamily}(\ref{primesieve-primesieve_set_num_threads}) to change the number of threads.

\end{list}
\ifpdf
\subsection*{\large{\textbf{primesieve{\_}count{\_}quintuplets}}\normalsize\hspace{1ex}\hrulefill}
\else
\subsection*{primesieve{\_}count{\_}quintuplets}
\fi
\label{primesieve-primesieve_count_quintuplets}
\index{primesieve{\_}count{\_}quintuplets}
\begin{list}{}{
\settowidth{\tmplength}{\textbf{Description}}
\setlength{\itemindent}{0cm}
\setlength{\listparindent}{0cm}
\setlength{\leftmargin}{\evensidemargin}
\addtolength{\leftmargin}{\tmplength}
\settowidth{\labelsep}{X}
\addtolength{\leftmargin}{\labelsep}
\setlength{\labelwidth}{\tmplength}
}
\item[\textbf{Declaration}\hfill]
\ifpdf
\begin{flushleft}
\fi
\begin{ttfamily}
function primesieve{\_}count{\_}quintuplets(start: UInt64; stop: UInt64): UInt64; cdecl; external LIB{\_}PRIMESIEVE name LIB{\_}FNPFX + 'primesieve{\_}count{\_}quintuplets';\end{ttfamily}

\ifpdf
\end{flushleft}
\fi

\par
\item[\textbf{Description}]
Count the prime quintuplets within the interval \textit{[start, stop]}.

By default all CPU cores are used, use \begin{ttfamily}primesieve{\_}set{\_}num{\_}threads\end{ttfamily}(\ref{primesieve-primesieve_set_num_threads}) to change the number of threads.

\end{list}
\ifpdf
\subsection*{\large{\textbf{primesieve{\_}count{\_}sextuplets}}\normalsize\hspace{1ex}\hrulefill}
\else
\subsection*{primesieve{\_}count{\_}sextuplets}
\fi
\label{primesieve-primesieve_count_sextuplets}
\index{primesieve{\_}count{\_}sextuplets}
\begin{list}{}{
\settowidth{\tmplength}{\textbf{Description}}
\setlength{\itemindent}{0cm}
\setlength{\listparindent}{0cm}
\setlength{\leftmargin}{\evensidemargin}
\addtolength{\leftmargin}{\tmplength}
\settowidth{\labelsep}{X}
\addtolength{\leftmargin}{\labelsep}
\setlength{\labelwidth}{\tmplength}
}
\item[\textbf{Declaration}\hfill]
\ifpdf
\begin{flushleft}
\fi
\begin{ttfamily}
function primesieve{\_}count{\_}sextuplets(start: UInt64; stop: UInt64): UInt64; cdecl; external LIB{\_}PRIMESIEVE name LIB{\_}FNPFX + 'primesieve{\_}count{\_}sextuplets';\end{ttfamily}

\ifpdf
\end{flushleft}
\fi

\par
\item[\textbf{Description}]
Count the prime sextuplets within the interval \textit{[start, stop]}.

By default all CPU cores are used, use \begin{ttfamily}primesieve{\_}set{\_}num{\_}threads\end{ttfamily}(\ref{primesieve-primesieve_set_num_threads}) to change the number of threads.

\end{list}
\ifpdf
\subsection*{\large{\textbf{primesieve{\_}print{\_}primes}}\normalsize\hspace{1ex}\hrulefill}
\else
\subsection*{primesieve{\_}print{\_}primes}
\fi
\label{primesieve-primesieve_print_primes}
\index{primesieve{\_}print{\_}primes}
\begin{list}{}{
\settowidth{\tmplength}{\textbf{Description}}
\setlength{\itemindent}{0cm}
\setlength{\listparindent}{0cm}
\setlength{\leftmargin}{\evensidemargin}
\addtolength{\leftmargin}{\tmplength}
\settowidth{\labelsep}{X}
\addtolength{\leftmargin}{\labelsep}
\setlength{\labelwidth}{\tmplength}
}
\item[\textbf{Declaration}\hfill]
\ifpdf
\begin{flushleft}
\fi
\begin{ttfamily}
procedure primesieve{\_}print{\_}primes(start: UInt64; stop: UInt64); cdecl; external LIB{\_}PRIMESIEVE name LIB{\_}FNPFX + 'primesieve{\_}print{\_}primes';\end{ttfamily}

\ifpdf
\end{flushleft}
\fi

\par
\item[\textbf{Description}]
Print the primes within the interval \textit{[start, stop]} to the standard output.

\end{list}
\ifpdf
\subsection*{\large{\textbf{primesieve{\_}print{\_}twins}}\normalsize\hspace{1ex}\hrulefill}
\else
\subsection*{primesieve{\_}print{\_}twins}
\fi
\label{primesieve-primesieve_print_twins}
\index{primesieve{\_}print{\_}twins}
\begin{list}{}{
\settowidth{\tmplength}{\textbf{Description}}
\setlength{\itemindent}{0cm}
\setlength{\listparindent}{0cm}
\setlength{\leftmargin}{\evensidemargin}
\addtolength{\leftmargin}{\tmplength}
\settowidth{\labelsep}{X}
\addtolength{\leftmargin}{\labelsep}
\setlength{\labelwidth}{\tmplength}
}
\item[\textbf{Declaration}\hfill]
\ifpdf
\begin{flushleft}
\fi
\begin{ttfamily}
procedure primesieve{\_}print{\_}twins(start: UInt64; stop: UInt64); cdecl; external LIB{\_}PRIMESIEVE name LIB{\_}FNPFX + 'primesieve{\_}print{\_}twins';\end{ttfamily}

\ifpdf
\end{flushleft}
\fi

\par
\item[\textbf{Description}]
Print the twin primes within the interval \textit{[start, stop]} to the standard output.

\end{list}
\ifpdf
\subsection*{\large{\textbf{primesieve{\_}print{\_}triplets}}\normalsize\hspace{1ex}\hrulefill}
\else
\subsection*{primesieve{\_}print{\_}triplets}
\fi
\label{primesieve-primesieve_print_triplets}
\index{primesieve{\_}print{\_}triplets}
\begin{list}{}{
\settowidth{\tmplength}{\textbf{Description}}
\setlength{\itemindent}{0cm}
\setlength{\listparindent}{0cm}
\setlength{\leftmargin}{\evensidemargin}
\addtolength{\leftmargin}{\tmplength}
\settowidth{\labelsep}{X}
\addtolength{\leftmargin}{\labelsep}
\setlength{\labelwidth}{\tmplength}
}
\item[\textbf{Declaration}\hfill]
\ifpdf
\begin{flushleft}
\fi
\begin{ttfamily}
procedure primesieve{\_}print{\_}triplets(start: UInt64; stop: UInt64); cdecl; external LIB{\_}PRIMESIEVE name LIB{\_}FNPFX + 'primesieve{\_}print{\_}triplets';\end{ttfamily}

\ifpdf
\end{flushleft}
\fi

\par
\item[\textbf{Description}]
Print the prime triplets within the interval \textit{[start, stop]} to the standard output.

\end{list}
\ifpdf
\subsection*{\large{\textbf{primesieve{\_}print{\_}quadruplets}}\normalsize\hspace{1ex}\hrulefill}
\else
\subsection*{primesieve{\_}print{\_}quadruplets}
\fi
\label{primesieve-primesieve_print_quadruplets}
\index{primesieve{\_}print{\_}quadruplets}
\begin{list}{}{
\settowidth{\tmplength}{\textbf{Description}}
\setlength{\itemindent}{0cm}
\setlength{\listparindent}{0cm}
\setlength{\leftmargin}{\evensidemargin}
\addtolength{\leftmargin}{\tmplength}
\settowidth{\labelsep}{X}
\addtolength{\leftmargin}{\labelsep}
\setlength{\labelwidth}{\tmplength}
}
\item[\textbf{Declaration}\hfill]
\ifpdf
\begin{flushleft}
\fi
\begin{ttfamily}
procedure primesieve{\_}print{\_}quadruplets(start: UInt64; stop: UInt64); cdecl; external LIB{\_}PRIMESIEVE name LIB{\_}FNPFX + 'primesieve{\_}print{\_}quadruplets';\end{ttfamily}

\ifpdf
\end{flushleft}
\fi

\par
\item[\textbf{Description}]
Print the prime quadruplets within the interval \textit{[start, stop]} to the standard output.

\end{list}
\ifpdf
\subsection*{\large{\textbf{primesieve{\_}print{\_}quintuplets}}\normalsize\hspace{1ex}\hrulefill}
\else
\subsection*{primesieve{\_}print{\_}quintuplets}
\fi
\label{primesieve-primesieve_print_quintuplets}
\index{primesieve{\_}print{\_}quintuplets}
\begin{list}{}{
\settowidth{\tmplength}{\textbf{Description}}
\setlength{\itemindent}{0cm}
\setlength{\listparindent}{0cm}
\setlength{\leftmargin}{\evensidemargin}
\addtolength{\leftmargin}{\tmplength}
\settowidth{\labelsep}{X}
\addtolength{\leftmargin}{\labelsep}
\setlength{\labelwidth}{\tmplength}
}
\item[\textbf{Declaration}\hfill]
\ifpdf
\begin{flushleft}
\fi
\begin{ttfamily}
procedure primesieve{\_}print{\_}quintuplets(start: UInt64; stop: UInt64); cdecl; external LIB{\_}PRIMESIEVE name LIB{\_}FNPFX + 'primesieve{\_}print{\_}quintuplets';\end{ttfamily}

\ifpdf
\end{flushleft}
\fi

\par
\item[\textbf{Description}]
Print the prime quintuplets within the interval \textit{[start, stop]} to the standard output.

\end{list}
\ifpdf
\subsection*{\large{\textbf{primesieve{\_}print{\_}sextuplets}}\normalsize\hspace{1ex}\hrulefill}
\else
\subsection*{primesieve{\_}print{\_}sextuplets}
\fi
\label{primesieve-primesieve_print_sextuplets}
\index{primesieve{\_}print{\_}sextuplets}
\begin{list}{}{
\settowidth{\tmplength}{\textbf{Description}}
\setlength{\itemindent}{0cm}
\setlength{\listparindent}{0cm}
\setlength{\leftmargin}{\evensidemargin}
\addtolength{\leftmargin}{\tmplength}
\settowidth{\labelsep}{X}
\addtolength{\leftmargin}{\labelsep}
\setlength{\labelwidth}{\tmplength}
}
\item[\textbf{Declaration}\hfill]
\ifpdf
\begin{flushleft}
\fi
\begin{ttfamily}
procedure primesieve{\_}print{\_}sextuplets(start: UInt64; stop: UInt64); cdecl; external LIB{\_}PRIMESIEVE name LIB{\_}FNPFX + 'primesieve{\_}print{\_}sextuplets';\end{ttfamily}

\ifpdf
\end{flushleft}
\fi

\par
\item[\textbf{Description}]
Print the prime sextuplets within the interval \textit{[start, stop]} to the standard output.

\end{list}
\ifpdf
\subsection*{\large{\textbf{primesieve{\_}get{\_}max{\_}stop}}\normalsize\hspace{1ex}\hrulefill}
\else
\subsection*{primesieve{\_}get{\_}max{\_}stop}
\fi
\label{primesieve-primesieve_get_max_stop}
\index{primesieve{\_}get{\_}max{\_}stop}
\begin{list}{}{
\settowidth{\tmplength}{\textbf{Description}}
\setlength{\itemindent}{0cm}
\setlength{\listparindent}{0cm}
\setlength{\leftmargin}{\evensidemargin}
\addtolength{\leftmargin}{\tmplength}
\settowidth{\labelsep}{X}
\addtolength{\leftmargin}{\labelsep}
\setlength{\labelwidth}{\tmplength}
}
\item[\textbf{Declaration}\hfill]
\ifpdf
\begin{flushleft}
\fi
\begin{ttfamily}
function primesieve{\_}get{\_}max{\_}stop(): UInt64; cdecl; external LIB{\_}PRIMESIEVE name LIB{\_}FNPFX + 'primesieve{\_}get{\_}max{\_}stop';\end{ttfamily}

\ifpdf
\end{flushleft}
\fi

\par
\item[\textbf{Description}]
Returns the largest valid stop number for primesieve.

\textit{2{\^{}}64{-}1 (UINT64{\_}MAX)}

\end{list}
\ifpdf
\subsection*{\large{\textbf{primesieve{\_}get{\_}sieve{\_}size}}\normalsize\hspace{1ex}\hrulefill}
\else
\subsection*{primesieve{\_}get{\_}sieve{\_}size}
\fi
\label{primesieve-primesieve_get_sieve_size}
\index{primesieve{\_}get{\_}sieve{\_}size}
\begin{list}{}{
\settowidth{\tmplength}{\textbf{Description}}
\setlength{\itemindent}{0cm}
\setlength{\listparindent}{0cm}
\setlength{\leftmargin}{\evensidemargin}
\addtolength{\leftmargin}{\tmplength}
\settowidth{\labelsep}{X}
\addtolength{\leftmargin}{\labelsep}
\setlength{\labelwidth}{\tmplength}
}
\item[\textbf{Declaration}\hfill]
\ifpdf
\begin{flushleft}
\fi
\begin{ttfamily}
function primesieve{\_}get{\_}sieve{\_}size(): Integer; cdecl; external LIB{\_}PRIMESIEVE name LIB{\_}FNPFX + 'primesieve{\_}get{\_}sieve{\_}size';\end{ttfamily}

\ifpdf
\end{flushleft}
\fi

\par
\item[\textbf{Description}]
Get the current set sieve size in KiB

\end{list}
\ifpdf
\subsection*{\large{\textbf{primesieve{\_}get{\_}num{\_}threads}}\normalsize\hspace{1ex}\hrulefill}
\else
\subsection*{primesieve{\_}get{\_}num{\_}threads}
\fi
\label{primesieve-primesieve_get_num_threads}
\index{primesieve{\_}get{\_}num{\_}threads}
\begin{list}{}{
\settowidth{\tmplength}{\textbf{Description}}
\setlength{\itemindent}{0cm}
\setlength{\listparindent}{0cm}
\setlength{\leftmargin}{\evensidemargin}
\addtolength{\leftmargin}{\tmplength}
\settowidth{\labelsep}{X}
\addtolength{\leftmargin}{\labelsep}
\setlength{\labelwidth}{\tmplength}
}
\item[\textbf{Declaration}\hfill]
\ifpdf
\begin{flushleft}
\fi
\begin{ttfamily}
function primesieve{\_}get{\_}num{\_}threads(): Integer; cdecl; external LIB{\_}PRIMESIEVE name LIB{\_}FNPFX + 'primesieve{\_}get{\_}num{\_}threads';\end{ttfamily}

\ifpdf
\end{flushleft}
\fi

\par
\item[\textbf{Description}]
Get the current set number of threads

\end{list}
\ifpdf
\subsection*{\large{\textbf{primesieve{\_}set{\_}sieve{\_}size}}\normalsize\hspace{1ex}\hrulefill}
\else
\subsection*{primesieve{\_}set{\_}sieve{\_}size}
\fi
\label{primesieve-primesieve_set_sieve_size}
\index{primesieve{\_}set{\_}sieve{\_}size}
\begin{list}{}{
\settowidth{\tmplength}{\textbf{Description}}
\setlength{\itemindent}{0cm}
\setlength{\listparindent}{0cm}
\setlength{\leftmargin}{\evensidemargin}
\addtolength{\leftmargin}{\tmplength}
\settowidth{\labelsep}{X}
\addtolength{\leftmargin}{\labelsep}
\setlength{\labelwidth}{\tmplength}
}
\item[\textbf{Declaration}\hfill]
\ifpdf
\begin{flushleft}
\fi
\begin{ttfamily}
procedure primesieve{\_}set{\_}sieve{\_}size(sieve{\_}size: Integer); cdecl; external LIB{\_}PRIMESIEVE name LIB{\_}FNPFX + 'primesieve{\_}set{\_}sieve{\_}size';\end{ttfamily}

\ifpdf
\end{flushleft}
\fi

\par
\item[\textbf{Description}]
Set the sieve size in KiB (kibibyte). The best sieving performance is achieved with a sieve size of your CPU's L1 or L2 cache size (per core). \textit{sieve{\_}size {$>$}= 8 and {$<$}= 4096}

\end{list}
\ifpdf
\subsection*{\large{\textbf{primesieve{\_}set{\_}num{\_}threads}}\normalsize\hspace{1ex}\hrulefill}
\else
\subsection*{primesieve{\_}set{\_}num{\_}threads}
\fi
\label{primesieve-primesieve_set_num_threads}
\index{primesieve{\_}set{\_}num{\_}threads}
\begin{list}{}{
\settowidth{\tmplength}{\textbf{Description}}
\setlength{\itemindent}{0cm}
\setlength{\listparindent}{0cm}
\setlength{\leftmargin}{\evensidemargin}
\addtolength{\leftmargin}{\tmplength}
\settowidth{\labelsep}{X}
\addtolength{\leftmargin}{\labelsep}
\setlength{\labelwidth}{\tmplength}
}
\item[\textbf{Declaration}\hfill]
\ifpdf
\begin{flushleft}
\fi
\begin{ttfamily}
procedure primesieve{\_}set{\_}num{\_}threads(num{\_}threads: Integer); cdecl; external LIB{\_}PRIMESIEVE name LIB{\_}FNPFX + 'primesieve{\_}set{\_}num{\_}threads';\end{ttfamily}

\ifpdf
\end{flushleft}
\fi

\par
\item[\textbf{Description}]
Set the number of threads for use in \textit{primesieve{\_}count{\_}*()} and \begin{ttfamily}primesieve{\_}nth{\_}prime\end{ttfamily}(\ref{primesieve-primesieve_nth_prime}). By default all CPU cores are used.

\end{list}
\ifpdf
\subsection*{\large{\textbf{primesieve{\_}free}}\normalsize\hspace{1ex}\hrulefill}
\else
\subsection*{primesieve{\_}free}
\fi
\label{primesieve-primesieve_free}
\index{primesieve{\_}free}
\begin{list}{}{
\settowidth{\tmplength}{\textbf{Description}}
\setlength{\itemindent}{0cm}
\setlength{\listparindent}{0cm}
\setlength{\leftmargin}{\evensidemargin}
\addtolength{\leftmargin}{\tmplength}
\settowidth{\labelsep}{X}
\addtolength{\leftmargin}{\labelsep}
\setlength{\labelwidth}{\tmplength}
}
\item[\textbf{Declaration}\hfill]
\ifpdf
\begin{flushleft}
\fi
\begin{ttfamily}
procedure primesieve{\_}free(primes: Pointer); cdecl; external LIB{\_}PRIMESIEVE name LIB{\_}FNPFX + 'primesieve{\_}free';\end{ttfamily}

\ifpdf
\end{flushleft}
\fi

\par
\item[\textbf{Description}]
Deallocate a primes array created using the \begin{ttfamily}primesieve{\_}generate{\_}primes\end{ttfamily}(\ref{primesieve-primesieve_generate_primes}) or \begin{ttfamily}primesieve{\_}generate{\_}n{\_}primes\end{ttfamily}(\ref{primesieve-primesieve_generate_n_primes}) functions.

\end{list}
\ifpdf
\subsection*{\large{\textbf{primesieve{\_}version}}\normalsize\hspace{1ex}\hrulefill}
\else
\subsection*{primesieve{\_}version}
\fi
\label{primesieve-primesieve_version}
\index{primesieve{\_}version}
\begin{list}{}{
\settowidth{\tmplength}{\textbf{Description}}
\setlength{\itemindent}{0cm}
\setlength{\listparindent}{0cm}
\setlength{\leftmargin}{\evensidemargin}
\addtolength{\leftmargin}{\tmplength}
\settowidth{\labelsep}{X}
\addtolength{\leftmargin}{\labelsep}
\setlength{\labelwidth}{\tmplength}
}
\item[\textbf{Declaration}\hfill]
\ifpdf
\begin{flushleft}
\fi
\begin{ttfamily}
function primesieve{\_}version(): PAnsiChar; cdecl; external LIB{\_}PRIMESIEVE name LIB{\_}FNPFX + 'primesieve{\_}version';\end{ttfamily}

\ifpdf
\end{flushleft}
\fi

\par
\item[\textbf{Description}]
Get the primesieve version number, in the form “i.j”

\end{list}
\ifpdf
\subsection*{\large{\textbf{primesieve{\_}init}}\normalsize\hspace{1ex}\hrulefill}
\else
\subsection*{primesieve{\_}init}
\fi
\label{primesieve-primesieve_init}
\index{primesieve{\_}init}
\begin{list}{}{
\settowidth{\tmplength}{\textbf{Description}}
\setlength{\itemindent}{0cm}
\setlength{\listparindent}{0cm}
\setlength{\leftmargin}{\evensidemargin}
\addtolength{\leftmargin}{\tmplength}
\settowidth{\labelsep}{X}
\addtolength{\leftmargin}{\labelsep}
\setlength{\labelwidth}{\tmplength}
}
\item[\textbf{Declaration}\hfill]
\ifpdf
\begin{flushleft}
\fi
\begin{ttfamily}
procedure primesieve{\_}init(var it: primesieve{\_}iterator); cdecl; external LIB{\_}PRIMESIEVE name LIB{\_}FNPFX + 'primesieve{\_}init';\end{ttfamily}

\ifpdf
\end{flushleft}
\fi

\par
\item[\textbf{Description}]
Initialize the primesieve iterator before first using it

\end{list}
\ifpdf
\subsection*{\large{\textbf{primesieve{\_}free{\_}iterator}}\normalsize\hspace{1ex}\hrulefill}
\else
\subsection*{primesieve{\_}free{\_}iterator}
\fi
\label{primesieve-primesieve_free_iterator}
\index{primesieve{\_}free{\_}iterator}
\begin{list}{}{
\settowidth{\tmplength}{\textbf{Description}}
\setlength{\itemindent}{0cm}
\setlength{\listparindent}{0cm}
\setlength{\leftmargin}{\evensidemargin}
\addtolength{\leftmargin}{\tmplength}
\settowidth{\labelsep}{X}
\addtolength{\leftmargin}{\labelsep}
\setlength{\labelwidth}{\tmplength}
}
\item[\textbf{Declaration}\hfill]
\ifpdf
\begin{flushleft}
\fi
\begin{ttfamily}
procedure primesieve{\_}free{\_}iterator(var it: primesieve{\_}iterator); cdecl; external LIB{\_}PRIMESIEVE name LIB{\_}FNPFX + 'primesieve{\_}free{\_}iterator';\end{ttfamily}

\ifpdf
\end{flushleft}
\fi

\par
\item[\textbf{Description}]
Free all iterator memory

\end{list}
\ifpdf
\subsection*{\large{\textbf{primesieve{\_}skipto}}\normalsize\hspace{1ex}\hrulefill}
\else
\subsection*{primesieve{\_}skipto}
\fi
\label{primesieve-primesieve_skipto}
\index{primesieve{\_}skipto}
\begin{list}{}{
\settowidth{\tmplength}{\textbf{Description}}
\setlength{\itemindent}{0cm}
\setlength{\listparindent}{0cm}
\setlength{\leftmargin}{\evensidemargin}
\addtolength{\leftmargin}{\tmplength}
\settowidth{\labelsep}{X}
\addtolength{\leftmargin}{\labelsep}
\setlength{\labelwidth}{\tmplength}
}
\item[\textbf{Declaration}\hfill]
\ifpdf
\begin{flushleft}
\fi
\begin{ttfamily}
procedure primesieve{\_}skipto(var it: primesieve{\_}iterator; start: UInt64; stop{\_}hint: UInt64); cdecl; external LIB{\_}PRIMESIEVE name LIB{\_}FNPFX + 'primesieve{\_}skipto';\end{ttfamily}

\ifpdf
\end{flushleft}
\fi

\par
\item[\textbf{Description}]
Reset the primesieve iterator to start.



\par
\item[\textbf{Parameters}]
\begin{description}
\item[start] Generate \textit{primes {$>$} start (or {$<$} start)}
\item[stop{\_}hint] Stop number optimization hint. E.g. if you want to generate the primes below \textit{1000} use \textit{stop{\_}hint = 1000}, if you don't know use \begin{ttfamily}primesieve{\_}get{\_}max{\_}stop\end{ttfamily}(\ref{primesieve-primesieve_get_max_stop})
\end{description}


\end{list}
\ifpdf
\subsection*{\large{\textbf{primesieve{\_}next{\_}prime}}\normalsize\hspace{1ex}\hrulefill}
\else
\subsection*{primesieve{\_}next{\_}prime}
\fi
\label{primesieve-primesieve_next_prime}
\index{primesieve{\_}next{\_}prime}
\begin{list}{}{
\settowidth{\tmplength}{\textbf{Description}}
\setlength{\itemindent}{0cm}
\setlength{\listparindent}{0cm}
\setlength{\leftmargin}{\evensidemargin}
\addtolength{\leftmargin}{\tmplength}
\settowidth{\labelsep}{X}
\addtolength{\leftmargin}{\labelsep}
\setlength{\labelwidth}{\tmplength}
}
\item[\textbf{Declaration}\hfill]
\ifpdf
\begin{flushleft}
\fi
\begin{ttfamily}
function primesieve{\_}next{\_}prime(var it: primesieve{\_}iterator): UInt64; inline;\end{ttfamily}

\ifpdf
\end{flushleft}
\fi

\par
\item[\textbf{Description}]
Get the next prime.

Returns \textit{UINT64{\_}MAX} if next \textit{prime {$>$} 2{\^{}}64}.

\end{list}
\ifpdf
\subsection*{\large{\textbf{primesieve{\_}prev{\_}prime}}\normalsize\hspace{1ex}\hrulefill}
\else
\subsection*{primesieve{\_}prev{\_}prime}
\fi
\label{primesieve-primesieve_prev_prime}
\index{primesieve{\_}prev{\_}prime}
\begin{list}{}{
\settowidth{\tmplength}{\textbf{Description}}
\setlength{\itemindent}{0cm}
\setlength{\listparindent}{0cm}
\setlength{\leftmargin}{\evensidemargin}
\addtolength{\leftmargin}{\tmplength}
\settowidth{\labelsep}{X}
\addtolength{\leftmargin}{\labelsep}
\setlength{\labelwidth}{\tmplength}
}
\item[\textbf{Declaration}\hfill]
\ifpdf
\begin{flushleft}
\fi
\begin{ttfamily}
function primesieve{\_}prev{\_}prime(var it: primesieve{\_}iterator): UInt64; inline;\end{ttfamily}

\ifpdf
\end{flushleft}
\fi

\par
\item[\textbf{Description}]
Get the previous prime.

\begin{ttfamily}primesieve{\_}prev{\_}prime\end{ttfamily}(\ref{primesieve-primesieve_prev_prime}) returns \textit{0} for \textit{n {$<$}= 2}. Note that \begin{ttfamily}primesieve{\_}next{\_}prime\end{ttfamily}(\ref{primesieve-primesieve_next_prime}) runs up to 2x faster than \begin{ttfamily}primesieve{\_}prev{\_}prime\end{ttfamily}(\ref{primesieve-primesieve_prev_prime}). Hence if the same algorithm can be written using either \begin{ttfamily}primesieve{\_}prev{\_}prime\end{ttfamily}(\ref{primesieve-primesieve_prev_prime}) or \begin{ttfamily}primesieve{\_}next{\_}prime\end{ttfamily}(\ref{primesieve-primesieve_next_prime}) it is preferable to use \begin{ttfamily}primesieve{\_}next{\_}prime\end{ttfamily}(\ref{primesieve-primesieve_next_prime}).

\end{list}
\section{Types}
\ifpdf
\subsection*{\large{\textbf{PUInt64}}\normalsize\hspace{1ex}\hrulefill}
\else
\subsection*{PUInt64}
\fi
\label{primesieve-PUInt64}
\index{PUInt64}
\begin{list}{}{
\settowidth{\tmplength}{\textbf{Description}}
\setlength{\itemindent}{0cm}
\setlength{\listparindent}{0cm}
\setlength{\leftmargin}{\evensidemargin}
\addtolength{\leftmargin}{\tmplength}
\settowidth{\labelsep}{X}
\addtolength{\leftmargin}{\labelsep}
\setlength{\labelwidth}{\tmplength}
}
\item[\textbf{Declaration}\hfill]
\ifpdf
\begin{flushleft}
\fi
\begin{ttfamily}
PUInt64 = {\^{}}UInt64;\end{ttfamily}

\ifpdf
\end{flushleft}
\fi

\end{list}
\ifpdf
\subsection*{\large{\textbf{PInt64}}\normalsize\hspace{1ex}\hrulefill}
\else
\subsection*{PInt64}
\fi
\label{primesieve-PInt64}
\index{PInt64}
\begin{list}{}{
\settowidth{\tmplength}{\textbf{Description}}
\setlength{\itemindent}{0cm}
\setlength{\listparindent}{0cm}
\setlength{\leftmargin}{\evensidemargin}
\addtolength{\leftmargin}{\tmplength}
\settowidth{\labelsep}{X}
\addtolength{\leftmargin}{\labelsep}
\setlength{\labelwidth}{\tmplength}
}
\item[\textbf{Declaration}\hfill]
\ifpdf
\begin{flushleft}
\fi
\begin{ttfamily}
PInt64 = {\^{}}Int64;\end{ttfamily}

\ifpdf
\end{flushleft}
\fi

\end{list}
\section{Constants}
\ifpdf
\subsection*{\large{\textbf{{\_}PRIMESIEVE{\_}VERSION}}\normalsize\hspace{1ex}\hrulefill}
\else
\subsection*{{\_}PRIMESIEVE{\_}VERSION}
\fi
\label{primesieve-_PRIMESIEVE_VERSION}
\index{{\_}PRIMESIEVE{\_}VERSION}
\begin{list}{}{
\settowidth{\tmplength}{\textbf{Description}}
\setlength{\itemindent}{0cm}
\setlength{\listparindent}{0cm}
\setlength{\leftmargin}{\evensidemargin}
\addtolength{\leftmargin}{\tmplength}
\settowidth{\labelsep}{X}
\addtolength{\leftmargin}{\labelsep}
\setlength{\labelwidth}{\tmplength}
}
\item[\textbf{Declaration}\hfill]
\ifpdf
\begin{flushleft}
\fi
\begin{ttfamily}
{\_}PRIMESIEVE{\_}VERSION = '7.5';\end{ttfamily}

\ifpdf
\end{flushleft}
\fi

\end{list}
\ifpdf
\subsection*{\large{\textbf{{\_}PRIMESIEVE{\_}VERSION{\_}MAJOR}}\normalsize\hspace{1ex}\hrulefill}
\else
\subsection*{{\_}PRIMESIEVE{\_}VERSION{\_}MAJOR}
\fi
\label{primesieve-_PRIMESIEVE_VERSION_MAJOR}
\index{{\_}PRIMESIEVE{\_}VERSION{\_}MAJOR}
\begin{list}{}{
\settowidth{\tmplength}{\textbf{Description}}
\setlength{\itemindent}{0cm}
\setlength{\listparindent}{0cm}
\setlength{\leftmargin}{\evensidemargin}
\addtolength{\leftmargin}{\tmplength}
\settowidth{\labelsep}{X}
\addtolength{\leftmargin}{\labelsep}
\setlength{\labelwidth}{\tmplength}
}
\item[\textbf{Declaration}\hfill]
\ifpdf
\begin{flushleft}
\fi
\begin{ttfamily}
{\_}PRIMESIEVE{\_}VERSION{\_}MAJOR = 7;\end{ttfamily}

\ifpdf
\end{flushleft}
\fi

\end{list}
\ifpdf
\subsection*{\large{\textbf{{\_}PRIMESIEVE{\_}VERSION{\_}MINOR}}\normalsize\hspace{1ex}\hrulefill}
\else
\subsection*{{\_}PRIMESIEVE{\_}VERSION{\_}MINOR}
\fi
\label{primesieve-_PRIMESIEVE_VERSION_MINOR}
\index{{\_}PRIMESIEVE{\_}VERSION{\_}MINOR}
\begin{list}{}{
\settowidth{\tmplength}{\textbf{Description}}
\setlength{\itemindent}{0cm}
\setlength{\listparindent}{0cm}
\setlength{\leftmargin}{\evensidemargin}
\addtolength{\leftmargin}{\tmplength}
\settowidth{\labelsep}{X}
\addtolength{\leftmargin}{\labelsep}
\setlength{\labelwidth}{\tmplength}
}
\item[\textbf{Declaration}\hfill]
\ifpdf
\begin{flushleft}
\fi
\begin{ttfamily}
{\_}PRIMESIEVE{\_}VERSION{\_}MINOR = 5;\end{ttfamily}

\ifpdf
\end{flushleft}
\fi

\end{list}
\ifpdf
\subsection*{\large{\textbf{{\_}PRIMESIEVE{\_}PAS{\_}VERSION}}\normalsize\hspace{1ex}\hrulefill}
\else
\subsection*{{\_}PRIMESIEVE{\_}PAS{\_}VERSION}
\fi
\label{primesieve-_PRIMESIEVE_PAS_VERSION}
\index{{\_}PRIMESIEVE{\_}PAS{\_}VERSION}
\begin{list}{}{
\settowidth{\tmplength}{\textbf{Description}}
\setlength{\itemindent}{0cm}
\setlength{\listparindent}{0cm}
\setlength{\leftmargin}{\evensidemargin}
\addtolength{\leftmargin}{\tmplength}
\settowidth{\labelsep}{X}
\addtolength{\leftmargin}{\labelsep}
\setlength{\labelwidth}{\tmplength}
}
\item[\textbf{Declaration}\hfill]
\ifpdf
\begin{flushleft}
\fi
\begin{ttfamily}
{\_}PRIMESIEVE{\_}PAS{\_}VERSION = '0.3';\end{ttfamily}

\ifpdf
\end{flushleft}
\fi

\par
\item[\textbf{Description}]
Pascal API version

\end{list}
\ifpdf
\subsection*{\large{\textbf{{\_}PRIMESIEVE{\_}ERROR}}\normalsize\hspace{1ex}\hrulefill}
\else
\subsection*{{\_}PRIMESIEVE{\_}ERROR}
\fi
\label{primesieve-_PRIMESIEVE_ERROR}
\index{{\_}PRIMESIEVE{\_}ERROR}
\begin{list}{}{
\settowidth{\tmplength}{\textbf{Description}}
\setlength{\itemindent}{0cm}
\setlength{\listparindent}{0cm}
\setlength{\leftmargin}{\evensidemargin}
\addtolength{\leftmargin}{\tmplength}
\settowidth{\labelsep}{X}
\addtolength{\leftmargin}{\labelsep}
\setlength{\labelwidth}{\tmplength}
}
\item[\textbf{Declaration}\hfill]
\ifpdf
\begin{flushleft}
\fi
\begin{ttfamily}
{\_}PRIMESIEVE{\_}ERROR = not UInt64(0);\end{ttfamily}

\ifpdf
\end{flushleft}
\fi

\par
\item[\textbf{Description}]
primesieve functions return \textit{PRIMESIEVE{\_}ERROR (UINT64{\_}MAX)} if any error occurs.

\end{list}
\ifpdf
\subsection*{\large{\textbf{INT16{\_}PRIMES}}\normalsize\hspace{1ex}\hrulefill}
\else
\subsection*{INT16{\_}PRIMES}
\fi
\label{primesieve-INT16_PRIMES}
\index{INT16{\_}PRIMES}
\begin{list}{}{
\settowidth{\tmplength}{\textbf{Description}}
\setlength{\itemindent}{0cm}
\setlength{\listparindent}{0cm}
\setlength{\leftmargin}{\evensidemargin}
\addtolength{\leftmargin}{\tmplength}
\settowidth{\labelsep}{X}
\addtolength{\leftmargin}{\labelsep}
\setlength{\labelwidth}{\tmplength}
}
\item[\textbf{Declaration}\hfill]
\ifpdf
\begin{flushleft}
\fi
\begin{ttfamily}
INT16{\_}PRIMES = 8;\end{ttfamily}

\ifpdf
\end{flushleft}
\fi

\par
\item[\textbf{Description}]
Generate primes of \textit{Int16 (c int16{\_}t)} type

\end{list}
\ifpdf
\subsection*{\large{\textbf{UINT16{\_}PRIMES}}\normalsize\hspace{1ex}\hrulefill}
\else
\subsection*{UINT16{\_}PRIMES}
\fi
\label{primesieve-UINT16_PRIMES}
\index{UINT16{\_}PRIMES}
\begin{list}{}{
\settowidth{\tmplength}{\textbf{Description}}
\setlength{\itemindent}{0cm}
\setlength{\listparindent}{0cm}
\setlength{\leftmargin}{\evensidemargin}
\addtolength{\leftmargin}{\tmplength}
\settowidth{\labelsep}{X}
\addtolength{\leftmargin}{\labelsep}
\setlength{\labelwidth}{\tmplength}
}
\item[\textbf{Declaration}\hfill]
\ifpdf
\begin{flushleft}
\fi
\begin{ttfamily}
UINT16{\_}PRIMES = 9;\end{ttfamily}

\ifpdf
\end{flushleft}
\fi

\par
\item[\textbf{Description}]
Generate primes of \textit{UInt16 (c uint16{\_}t)} type

\end{list}
\ifpdf
\subsection*{\large{\textbf{INT32{\_}PRIMES}}\normalsize\hspace{1ex}\hrulefill}
\else
\subsection*{INT32{\_}PRIMES}
\fi
\label{primesieve-INT32_PRIMES}
\index{INT32{\_}PRIMES}
\begin{list}{}{
\settowidth{\tmplength}{\textbf{Description}}
\setlength{\itemindent}{0cm}
\setlength{\listparindent}{0cm}
\setlength{\leftmargin}{\evensidemargin}
\addtolength{\leftmargin}{\tmplength}
\settowidth{\labelsep}{X}
\addtolength{\leftmargin}{\labelsep}
\setlength{\labelwidth}{\tmplength}
}
\item[\textbf{Declaration}\hfill]
\ifpdf
\begin{flushleft}
\fi
\begin{ttfamily}
INT32{\_}PRIMES = 10;\end{ttfamily}

\ifpdf
\end{flushleft}
\fi

\par
\item[\textbf{Description}]
Generate primes of \textit{Int32 (c int32{\_}t)} type

\end{list}
\ifpdf
\subsection*{\large{\textbf{UINT32{\_}PRIMES}}\normalsize\hspace{1ex}\hrulefill}
\else
\subsection*{UINT32{\_}PRIMES}
\fi
\label{primesieve-UINT32_PRIMES}
\index{UINT32{\_}PRIMES}
\begin{list}{}{
\settowidth{\tmplength}{\textbf{Description}}
\setlength{\itemindent}{0cm}
\setlength{\listparindent}{0cm}
\setlength{\leftmargin}{\evensidemargin}
\addtolength{\leftmargin}{\tmplength}
\settowidth{\labelsep}{X}
\addtolength{\leftmargin}{\labelsep}
\setlength{\labelwidth}{\tmplength}
}
\item[\textbf{Declaration}\hfill]
\ifpdf
\begin{flushleft}
\fi
\begin{ttfamily}
UINT32{\_}PRIMES = 11;\end{ttfamily}

\ifpdf
\end{flushleft}
\fi

\par
\item[\textbf{Description}]
Generate primes of \textit{UInt32 (c uint32{\_}t)} type

\end{list}
\ifpdf
\subsection*{\large{\textbf{INT64{\_}PRIMES}}\normalsize\hspace{1ex}\hrulefill}
\else
\subsection*{INT64{\_}PRIMES}
\fi
\label{primesieve-INT64_PRIMES}
\index{INT64{\_}PRIMES}
\begin{list}{}{
\settowidth{\tmplength}{\textbf{Description}}
\setlength{\itemindent}{0cm}
\setlength{\listparindent}{0cm}
\setlength{\leftmargin}{\evensidemargin}
\addtolength{\leftmargin}{\tmplength}
\settowidth{\labelsep}{X}
\addtolength{\leftmargin}{\labelsep}
\setlength{\labelwidth}{\tmplength}
}
\item[\textbf{Declaration}\hfill]
\ifpdf
\begin{flushleft}
\fi
\begin{ttfamily}
INT64{\_}PRIMES = 12;\end{ttfamily}

\ifpdf
\end{flushleft}
\fi

\par
\item[\textbf{Description}]
Generate primes of \textit{Int64 (c int64{\_}t)} type

\end{list}
\ifpdf
\subsection*{\large{\textbf{UINT64{\_}PRIMES}}\normalsize\hspace{1ex}\hrulefill}
\else
\subsection*{UINT64{\_}PRIMES}
\fi
\label{primesieve-UINT64_PRIMES}
\index{UINT64{\_}PRIMES}
\begin{list}{}{
\settowidth{\tmplength}{\textbf{Description}}
\setlength{\itemindent}{0cm}
\setlength{\listparindent}{0cm}
\setlength{\leftmargin}{\evensidemargin}
\addtolength{\leftmargin}{\tmplength}
\settowidth{\labelsep}{X}
\addtolength{\leftmargin}{\labelsep}
\setlength{\labelwidth}{\tmplength}
}
\item[\textbf{Declaration}\hfill]
\ifpdf
\begin{flushleft}
\fi
\begin{ttfamily}
UINT64{\_}PRIMES = 13;\end{ttfamily}

\ifpdf
\end{flushleft}
\fi

\par
\item[\textbf{Description}]
Generate primes of \textit{UInt64 (c uint64{\_}t)} type

\end{list}
\end{document}
